\documentclass{ctexart}
\usepackage{indentfirst}
\usepackage{lmodern}% http://ctan.org/pkg/lm
\usepackage{setspace}
\usepackage{verbatim}
\usepackage{amsmath}%数学
\usepackage{amsthm}%数学定理、证明
\usepackage{graphicx}%graphicx基于graphics,更方便
% \usepackage{subfig}%图像
\usepackage{subfigure}
\usepackage{booktabs}%表格
\usepackage{tabularx}%表格
\usepackage{multirow}
\usepackage{multicol}
\usepackage{listings}
\usepackage{relsize}
\usepackage[usenames]{color}   
%\usepackage{colortbl}%表格颜色
\usepackage[table]{xcolor}
\usepackage{array}
\usepackage[rightcaption]{sidecap}
\usepackage{hyperref}
\usepackage{float}

\hypersetup{
colorlinks=true,
linkcolor=black
}
\lstset{
         language = vhdl, numbers=left, 
         numberstyle=\tiny,keywordstyle=\color{blue!70},
         commentstyle=\color{red!50!green!50!blue!50},frame=shadowbox,
         rulesepcolor=\color{red!20!green!20!blue!20},basicstyle=\ttfamily
}
\renewcommand{\arraystretch}{1.8}%The height of each row, relative to its default height
\renewcommand{\figurename}{Fig.} %重定义编号前缀词
\renewcommand{\thesubfigure}{(\roman{subfigure})}%此外,还可设置图编号显示格式,加括号或者不加括号
\makeatletter \renewcommand{\@thesubfigure}{\thesubfigure \space}%子图编号与名称的间隔设置
\renewcommand{\p@subfigure}{} \makeatother

\setlength{\parindent}{2em}
\addtolength{\parskip}{3pt}

\onehalfspacing %1.5倍行距

\graphicspath{{./figures/}} 
\title{Hello World}
\begin{document}
\begin{titlepage}
        \vspace*{-2.5cm}
	
	\begin{figure}[h]
		\centering
		\includegraphics[width=0.7\linewidth]{zjdx}
	\end{figure}

	\begin{figure}[h]
		\centering
		\includegraphics[width=0.5\linewidth]{QSY}
	\end{figure}
	\vspace{-0.5cm}
	\begin{center}
		\Huge{\textbf{2022\ CSM暑期实习}}\\
		
		\Huge{\textbf{实习报告}}
	\end{center}
	
	
	\vspace*{1.5cm}

    \begin{center}
    \Large
            实验地点\ \ \underline{\makebox[220pt]{}}\\
            \vspace{0.3cm}
            \quad\ 姓\; 名 \ \ \underline{\makebox[220pt]{name}}\\
            \vspace{0.3cm}
            \quad\ 学\; 号\ \ \underline{\makebox[220pt]{123}}\\
            \vspace{0.3cm}
            实验日期\ \ \underline{\makebox[220pt]{2022年 9 月 18 日}}\\
            \vspace{0.3cm}
            指导老师\ \ \underline{\makebox[220pt]{name}}\\
            \vspace{0.3cm}
            \quad\ 成\; 绩\ \ \underline{\makebox[220pt]{114514}}\\

              
            

             
    \end{center}
        
    
\end{titlepage}
\end{document}
