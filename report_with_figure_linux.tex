%!TEX program = xelatex
\documentclass{ctexart}
\usepackage{indentfirst}
\usepackage{lmodern}% http://ctan.org/pkg/lm
\usepackage{setspace}
\usepackage{verbatim}
\usepackage{amsmath}%数学
\usepackage{amsthm}%数学定理、证明
\usepackage{graphicx}%graphicx基于graphics,更方便
% \usepackage{subfig}%图像
\usepackage{subfigure}
\usepackage{booktabs}%表格
\usepackage{tabularx}%表格
\usepackage{multirow}
\usepackage{multicol}
\usepackage{listings}
\usepackage{relsize}
\usepackage[usenames]{color}   
%\usepackage{colortbl}%表格颜色
\usepackage[table]{xcolor}
\usepackage{array}
\usepackage[rightcaption]{sidecap}
\usepackage{hyperref}
\usepackage{float}

\hypersetup{
colorlinks=true,
linkcolor=black
}
\lstset{
    basicstyle          =   \sffamily,          % 基本代码风格
    keywordstyle        =   \bfseries,          % 关键字风格
    commentstyle        =   \rmfamily\itshape,  % 注释的风格,斜体
    stringstyle         =   \ttfamily,  % 字符串风格
    flexiblecolumns,                % 别问为什么,加上这个
    numbers             =   left,   % 行号的位置在左边
    showspaces          =   false,  % 是否显示空格,显示了有点乱,所以不现实了
    numberstyle         =   \zihao{-5}\ttfamily,    % 行号的样式,小五号,tt等宽字体
    showstringspaces    =   false,
    captionpos          =   t,      % 这段代码的名字所呈现的位置,t指的是top上面
    frame               =   lrtb,   % 显示边框
}

\lstdefinestyle{Python}{
    language        =   Python, 
    basicstyle      =   \zihao{-5}\ttfamily,
    numberstyle     =   \zihao{-5}\ttfamily,
    keywordstyle    =   \color{blue},
    keywordstyle    =   [2] \color{teal},
    stringstyle     =   \color{magenta},
    commentstyle    =   \color{red}\ttfamily,
    breaklines      =   true,   
    columns         =   fixed,  
    basewidth       =   0.5em,
}
\renewcommand{\arraystretch}{1.8}%The height of each row, relative to its default height
\renewcommand{\figurename}{Fig.} %重定义编号前缀词
\renewcommand{\thesubfigure}{(\roman{subfigure})}%此外,还可设置图编号显示格式,加括号或者不加括号
\makeatletter \renewcommand{\@thesubfigure}{\thesubfigure \space}%子图编号与名称的间隔设置
\renewcommand{\p@subfigure}{} \makeatother

\setlength{\parindent}{2em}
\addtolength{\parskip}{3pt}

\onehalfspacing %1.5倍行距

\graphicspath{{./figures/}} 
\title{Hello World}
\begin{document}
\begin{titlepage}
        \vspace*{-2.5cm}
	
	\begin{figure}[h]
		\centering
		\includegraphics[width=0.7\linewidth]{zjdx}
	\end{figure}

	\begin{figure}[h]
		\centering
		\includegraphics[width=0.5\linewidth]{QSY}
	\end{figure}
	\vspace{-0.5cm}
	\begin{center}
		\Huge{\textbf{2022\ 暑期实习}}\\
		
		\Huge{\textbf{实习报告}}
	\end{center}
	
	
	\vspace*{1.5cm}

    \begin{center}
    \Large
            实验地点\ \ \underline{\makebox[220pt]{RCP Lab, CSM, United States}}\\
            \vspace{0.3cm}
            \quad\ 姓\; 名 \ \ \underline{\makebox[220pt]{Shenyao Jin}}\\
            \vspace{0.3cm}
            \quad\ 学\; 号\ \ \underline{\makebox[220pt]{}}\\
            \vspace{0.3cm}
            指导老师\ \ \underline{\makebox[220pt]{}}\\
            \vspace{0.3cm}
            \quad\ 成\; 绩\ \ \underline{\makebox[220pt]{}}\\
         
    \end{center}
        
    
\end{titlepage}

\newpage
% \setcounter{tocdepth}{5}
\tableofcontents
\thispagestyle{empty}%Removes the page numbering.
% \listoffigures%生成插图
% \listoftables%生成表格
\thispagestyle{empty}%Removes the page numbering.

\newpage
\pagenumbering{arabic}%重新开始标号,阿拉伯数字形式

\begin{thebibliography}{99} 
        \bibitem{ref1}Jin, G., Gaherty, J.B., Abers, G.A., Kim, Y., Eilon, Z. and Buck, W.R., 2015. Crust and upper mantle structure associated with extension in the W oodlark R ift, P apua N ew G uinea from R ayleigh‐wave tomography. Geochemistry, Geophysics, Geosystems, 16(11), pp.3808-3824.
        \bibitem{ref2}Feo, G., Sharma, J., Kortukov, D., Williams, W. and Ogunsanwo, T., 2020. Distributed fiber optic sensing for real-time monitoring of gas in riser during offshore drilling. Sensors, 20(1), p.267.
        \bibitem{ref3}Trnkoczy, A., 2009. Understanding and parameter setting of STA/LTA trigger algorithm. In New Manual of Seismological Observatory Practice (NMSOP) (pp. 1-20). Deutsches GeoForschungsZentrum GFZ.
\end{thebibliography}
\end{document}
